\documentclass[10pt,a4paper]{article}
\title{\textbf{P}rojet \textbf{F}inal : \textbf{A}ccess}
\author{
  ANAIS, BERTRAND - - VIEUX - MELCHIOR\\
  \texttt{anais.bertrand-----vieux---melchior@etu.univ-lyon1.fr}
  \and
  PRISCILLA, GOGUY\\
  \texttt{priscilla.goguy@etu.univ-lyon1.fr}
  \and
  MAHLÎ, REINETTE\\
  \texttt{mahli.reinette@etu.univ-lyon1.fr}
}
\date{13/10/2025}
\usepackage[utf8]{inputenc}
\usepackage[T1]{fontenc}
\usepackage[french]{babel}
\usepackage{graphicx}
\usepackage{hyperref}

\usepackage{listings}
\usepackage{xcolor}


\usepackage[many]{tcolorbox}
\usepackage{lipsum}
\usepackage{tikz}
\usetikzlibrary{automata, positioning, arrows}
\usepackage{pgfplots}
\usepackage{xcolor}
\usepackage{amsfonts}
\usepackage{dsfont}
\usepackage{xr-hyper} 
\usepackage{hyperref} 
\usepackage{algpseudocode}
\usepackage{algorithm}
\externaldocument[B-]{docB}[Annexe01.pdf]% <- full or relative path


% le préambule











% initialisation des couleurs
\definecolor{codegreen}{rgb}{0,0.6,0}
\definecolor{codegray}{rgb}{0.5,0.5,0.5}
\definecolor{codepurple}{rgb}{0.58,0,0.82}
\definecolor{backcolour}{rgb}{0.95,0.95,0.92}
\definecolor{mymauve}{rgb}{0.58,0,0.82}






%block de code 
\lstdefinestyle{mystyle}{
    backgroundcolor=\color{backcolour},   
    commentstyle=\color{codegreen},
    keywordstyle=\color{magenta},
    numberstyle=\tiny\color{codegray},
    stringstyle=\color{codepurple},
    basicstyle=\ttfamily\footnotesize,
    breakatwhitespace=false,         
    breaklines=true,                 
    captionpos=b,                    
    keepspaces=true,                 
    numbers=left,                    
    numbersep=5pt,                  
    showspaces=false,                
    showstringspaces=false,
    showtabs=false,                  
    tabsize=2
}

\lstset{style=mystyle}






%initialisation graphics
\tikzset{
->,
node distance = 2cm}



















% le corps du document






% le préambule
\begin{document}
%titre
\pagecolor{black!20}
\maketitle


\begin{center}
\textbf{\textit{M1 Actuariat}}

\textbf{\textit{ISFA}}

\includegraphics[width=4cm,height=2cm]{img1}

%\href{https://github.com/LaboiteNoire/techniques-de-simulations-}{\includegraphics[width=1cm,height=1cm]{img2}}

\end{center}


\newpage

\phantom{aaaaaa}

\tableofcontents

\newpage


\section{Préambule}
\subsection{Préambule}

\section{Construction}
\subsection{1. Source des données}
\subsection{2. Table Choix}
\subsection{3. Requêtes Lignes de départ et Lignes d'arrivée}
\subsection{4. Trajets directs}
\subsection{5. Correspondances}
\subsection{6. Trajets avec changements}
\subsection{7. Requêtes d'ajout, table des trajets}
\subsection{8. Interface utilisateur}
\subsection{9. Sous-formulaire}
\subsection{10. Compléments libres}
\section{Conclusions et Commentaires}
\subsection{Commentaires}
\subsection{Conclusions}







\end{document}